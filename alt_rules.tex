% Created 2023-10-16 Mo 22:25
% Intended LaTeX compiler: pdflatex
\documentclass[11pt]{article}
\usepackage[utf8]{inputenc}
\usepackage[T1]{fontenc}
\usepackage{graphicx}
\usepackage{longtable}
\usepackage{wrapfig}
\usepackage{rotating}
\usepackage[normalem]{ulem}
\usepackage{amsmath}
\usepackage{amssymb}
\usepackage{capt-of}
\usepackage{hyperref}
\usepackage[a5paper, total={128mm, 190mm}]{geometry}
\usepackage{multicol}
\setlength{\parindent}{0pt}
\setlength{\itemsep}{0.mm}
\usepackage{enumitem}
\setlist[itemize]{noitemsep}
\usepackage[table]{xcolor}
\usepackage[type={CC},modifier={by-sa}, version={4.0}, imagewidth=5em]{doclicense}
\renewcommand{\familydefault}{\sfdefault}
\makeatletter
\usepackage[explicit]{titlesec}
\titleformat{name=\section,numbered}[block]{\normalfont\Large\bfseries}{}{0em}{\colorbox{black!100}{ {\color{white}\thesection\quad #1} }}
\titleformat{name=\subsection}{\normalfont\large\bfseries}{}{0em}{\colorbox{black!66}{ {\color{white}\thesubsection\quad #1} }}
\titleformat{name=\subsubsection}{\normalfont\normalsize\bfseries}{}{0em}{\colorbox{black!33}{ {\color{black}\thesubsubsection\quad #1} }}
\makeatother
\renewcommand\maketitle{
\begin{titlepage}
\centering
\topskip300pt\vspace{5cm}
\fboxsep2em\colorbox{black!100}{
{\color{white}\bfseries\fontsize{24pt}{29pt}\selectfont \quad Primordial Waters\quad\par}
}
\vfill
a game by\par
\textsc{Lukas Zumvorde}

\vfill

% Bottom of the page
{\large \today\par}
\end{titlepage}}
\usepackage{xparse}
\usepackage{soul}
\setcounter{secnumdepth}{3}
\date{}
\title{Primordial Waters}
\hypersetup{
 pdfauthor={Lukas Zumvorde},
 pdftitle={Primordial Waters},
 pdfkeywords={},
 pdfsubject={},
 pdfcreator={Emacs 29.1 (Org mode 9.6.6)}, 
 pdflang={English}}
\begin{document}

\maketitle
\tableofcontents

{\rowcolors{1}{grey!20}{grey!10}


\section{Introduction}
\label{sec:org01c9883}

Primordial waters is a generic, light, and abstract role playing rule set with a focus on keeping the game balanced while maintaining and even promoting creative story telling.

\subsection{Goals}
\label{sec:org6c02b89}

\paragraph*{Rules Light}
\label{sec:orgb4802c9}

The intend is for the rules to be as small as possible given the other goals. It sould be easy to get into the game and easy to check the rules if you are not sure how a situation should be handled.

\paragraph*{World Agnostic}
\label{sec:orge38e12b}

The game should work with any game world or type of story resonably well.

\paragraph*{Scalable characters}
\label{sec:org522e34b}

It should be possible to play both superhumans and commoners. It should also be possible to use the same rules to display rats vs humans or humans vs. giant spaceships. Imagine a gian space battle taking place outside and the group of players running around as mere humans performing acts of sabotage to tip the battle in their sides favor. 

\paragraph*{No action economy}
\label{sec:orgb179f6b}

It should be possible to do multiple things simultaniously in a round. It should be a tradeoff. This adds a new dimension of interresting decision making to the game.

\paragraph*{Constistent propability distribution}
\label{sec:org9f3b9a2}

No matter how powerfull a character is, The propability distribution for his die rolls should be consistent and allow for interresting scenarios to occur. It should still feel fair.

\paragraph*{Easy Preparation}
\label{sec:org8e6e66b}

The game should allow GMs to prepare new characters and challenges easily and quickly. It should get out of his way, but still support him in keeping the games balance.

\paragraph*{Creative Character Build}
\label{sec:orga8ed255}

Players should be able to create nearly any character they like. It should also be possible to change the character at any point in the campagnie to reflect the characters development. In this the rules should invite creativity and just ensure the games balance.

\paragraph*{Support interresting storys}
\label{sec:org1f75cc2}

The rules should support the creation of interresting and consistent storys by providing anker points for unexpected results.

\paragraph*{Keep Game Balance}
\label{sec:org62f18d3}

In order for all to enjoy the game it must feel fair. Player characters need to feel similarly usefull and encounters with NPCs or challenges must feel beatable but challenging.

\paragraph*{Rules Out of your way}
\label{sec:orgf2f5bf0}

The rules should be there to help you keep the games balance, but they should not prevent you from creating the story you want to create.

\paragraph*{No extensive Bookkeeping}
\label{sec:orgd5247dd}

The game should be playable without extensive bookkeeping. IT should not be necessary to fill out a multiple pages long character sheet. During the game it should not be necessary to calculate or evaluate many values.

\subsection{Insipations}
\label{sec:orgceafffe}

This game did not come from nothing. Many other games had an impact on its creation. Some of the most impactfull ones are the following.

This games systems for aspects, fate points and movement were heavily inspired from the Fate Core system.
The distribution of attributes into categories was inspired by the AERA RPG.
The fate die to create unexpected events was inspired by the One Page Solo RPG Engine.


\subsection{How to use the game}
\label{sec:org7b63f30}

First rule of Gaming: Have fun. If the rules hinder you from having fun then screw the rules.


Use the aspects value as a guideline and be creative with what it could mean. Does a flame wall having a value of 4 mean that it has a difficulty of 4 to jump over it or does it attack anyone juming over with 4 dice? Do what fits your story. 

\newpage
\section{Rules}
\label{sec:org4ec6d18}

\subsection{Attributes}
\label{sec:org1985c5b}
Attributes describe a characters potential. The higher the value the greater things a character can achieve. There are the following 8 Attributes belonging to the 4 categories.

\begin{center}
\begin{tabular}{lll}
\textbf{Category} & \textbf{Attribute} & \textbf{Description}\\[0pt]
\hline
Physical & Strength & strength and hardiness\\[0pt]
 & Dexterity & agility, speed, precision\\[0pt]
\hline
Mental & Will & perseverance, attention\\[0pt]
 & Intellect & intelligence, knowledge\\[0pt]
\hline
Social & Empathy & understanding people\\[0pt]
 & Charisma & interacting with people\\[0pt]
\hline
Resources & Gear & Gear you have prepared\\[0pt]
 & Finances & Money and investments\\[0pt]
\end{tabular}
\end{center}

They determine the amount of dice you can use for \hyperref[sec:org06c2956]{checks}. 

\subsection{Aspects}
\label{sec:orgd71b72b}
Aspects are a combination of a descriptors and a value. When invoked the aspects value is added or subtracted from the number of dice rolled for \hyperref[sec:org06c2956]{checks}. An aspect is always bound to something. Be it a player, a NPC, an object or even a zone. Be creative with aspects. Try to bind their effects to the value and they should stay balanced enough.

\paragraph*{Creating an Aspect}
\label{sec:orgfb4830c}

An aspect can be created at any time by the GM. It can also be created by any player, npc or even object. To create an aspect all but the GM need to make a check. The aspect points (AP) from this check can be used to create an aspect. Increasing an aspects value by 1 costs 1 point.

\begin{quote}
Arthur the mage wants to enflame the gras around him to channel the enemies into a one on one battle with his Companions. The GM likes this idea and creates the aspect "Dry Grass 1" that Arthur can use. Arthur decides to cast his spell, succeeds with 3 AP and creates the aspect "Wall of Fire 3". Now any enemy has to either go around the wall or get burned. If an enemy tires to jumo through they have to roll on it against the Fires 3 dice. If the fire succeeds they get burned.
\end{quote}

\paragraph*{Using Aspects}
\label{sec:org55fc145}

Whenever it makes narative sense an aspect can be used. When used an aspect adds its value to the number of points that can be used for a check. It is also possible to call out an aspect to be used to hinder a check. Then the value is subtracted from the points instead. An aspect can both be used for actions and reactions but never at the same time.

\paragraph*{Area Aspects}
\label{sec:org6fc63c1}

An aspect can affect a zone. This means it can be invoked for all in the zone. To create such an aspect costs double the points. Exceptions may be made by the GM. 

\paragraph*{Multiple Tagets}
\label{sec:org8e73fd4}

If an aspect impacts multiple targets it costs double. Exceptions may be made by the GM.

\paragraph*{Duration of Aspects}
\label{sec:orgc0b7a69}

An aspect can be over within a round or it can last longer. Normaly aspects last at least a scene long. If your aspect should last longer it should cost double. If it should only last for a round the costs half. Exceptions may be made by the GM. In general it can be said, that an aspect lasts as long as it makes sense.

\paragraph*{Acting Aspects}
\label{sec:org61ce6b4}

Sometimes an aspect should perform actions on its own. Each round they can act like any other player or NPC. They use their value for their actions.

\begin{quote}
Poison Cloud 2: Each round it tries to use 2 dice to poison any inside. Since it impacts a zone it can do this for everyone inside the zone every round.
\end{quote}

\paragraph*{Resistant Aspects}
\label{sec:orgc2d0d14}

An aspect can have resistance to being removed. The resistance is subtracted from any attempt to removing this aspect. For each point of resistance costs 1. The effective resistance can never be more than the aspects value. Be creative when invoking a resistant aspect. For example armor could be implemented as an aspect with resistance.

\begin{quote}
A heavy door blocks the way. The GM creates the aspect "Fortified Door 5[2]". This means the aspect has a value of 5 with a resistance of 2. The players try to break through with pure strength. On their first try they get 4 AP. 2 get absorbed by the doors resistance and the rest weakens the door. The door s aspect is now "Fortified Door 3[2]". On their second try the players get 6 AP and break throgh.
If the players had decided to pick the lock the GM may had ignored the restistance value.
\end{quote}


\paragraph*{Character Aspects}
\label{sec:org30b6ee2}

Aspects can also be bound to a character and be bought with CP. If an aspect is mostly negative in nature it may give you points instead.

\paragraph*{Damage}
\label{sec:org5e5ca74}

Aspects are also used to represent damage a character or object has taken. If a character is unable to act in a scene because too many aspects hinder him, it is a good idea to narate them out of the scene. It is also a good time to create a fitting negative character aspect representing this loss (like "lost an arm" or "fear of water" or "hatefull of orks").


\subsection{Checks}
\label{sec:org06c2956}
Checks are rolled when the outcome of an action is not certain. Each check is bound to an attribute.
One gets a number of points equal to the sum off those from the attribute and applicable aspects. Distribute these point accoring to the rules.
\begin{itemize}
\item You can never roll more than 3 dice (plus the one fate die).
\item For N points roll one die that gives you N aspect points (AP) on a success (4, 5, or 6 eyes on the die).
\item All dice must be worth the same.
\item For 1 point reroll 1 die. At most you can reroll the number of dice you roll (maximum 3). You can use the reroll on the fate die.
\end{itemize}
If the aspect points are negative or 0 the check fails. 


You have P points from you attribute and aspects. 
Roll D dice:
\begin{itemize}
\item each die showing 4,5, or 6 is counted as a success
\item You can reroll (P modulo D) dice (divide P by D and take the rest)
\item 
\end{itemize}


\paragraph*{Difficulty}
\label{sec:orgfae05ae}
Difficulty reduces the number of aspect points. A good difficulty is N with N being the typical value of a sucessfull die roll. Reduce this to < N for easy checks and >=2xN for very hard checks.

\paragraph*{Collaborative checks:}
\label{sec:orgaa8f5be}
Everyone rolls individually and then sum together all successes.

\paragraph*{Risky checks:}
\label{sec:orgd0bc9b7}
If a check is risky the character might incur something bad if the check fails. If the check fails a negative aspect is created worth the difficulty in points. The minimum is 1. 

\paragraph*{Unexpected Results}
\label{sec:orgcb8aa4d}
You roll an additional d6, called the fate die. On a 1 you add a "but .." and create an additional aspect worth N points that counteracts the result somewhat. On a 6 you add a "and .." and create an additional aspect worth N points that enhances the result somewhat. The extra die can also be rerolled with a fate point or advantage on the check.

\begin{quote}
"Success and" during a fight against a goblin. You decapitate the goblin in an intimidating display, Not only does the goblin die but the display also weakens the goblins resolve. Likely they will try to flee after seeing this.
Aspect: Intimidating display 2.
\end{quote}

\begin{quote}
"Success but" during a fight against a goblin. You kill the goblin but are now covered in his blood, This has no effect on the fight itself but it may hinder any piece negotiation or help you when intimidating the remaining enemies.
Aspect: Covered in Blood 2.
\end{quote}

\paragraph*{Taking Time}
\label{sec:org0b881f1}
Sometimes a check is to difficult to achieve something within 1 check. Then it may be possible to do multiple checks over a longer time to accumulate the points needed. quickly. However you must decide beforehand how many checks you want to take. The AP of all checks are accumulated before considering the damage.

\subsection{Contest}
\label{sec:org8a6613b}
The prototipical contest is combat, but the same rules can be used for many other scenarios as well. A debate for example.

A contest is divided into rounds. Each participant in the contest can make one or more actions each round. When it is a participants turn or on any later point in the round they can perform an action.

\paragraph*{Actions}
\label{sec:org7dd7a18}
An action is a check that tries to create an aspect. A typical aspect would be to wound an enemy but it can be anything.

\paragraph*{Reactions}
\label{sec:org9be2579}
Whenever someone takes a action and has rolled his dice anyone else can immediately try to perform a reaction to prevent it. A reacton does not by its nature create an aspect. 

\paragraph*{Turn Order}
\label{sec:org1652a3c}
The participants take turn from the one with the highest relevant attribute (+ evtl aspects) to the lowest. On your turn you dont have to act. You can act at any point after you trun in the turn order. Even multiple times. 

\paragraph*{Multiple (re)actions*}
\label{sec:orgacd3246}
Each round you can take multiple actions and reactions. The total number of points gained from the attributes is the largest attribute value of the checks. From each attribute you can use at most its value in points in total.

\paragraph*{Acting together}
\label{sec:orgacf4577}
When acting together all values are combined and a single combined check is made or alternatively only the AP are combined. To act together all have to act at the same time in the turn order, so effectively at the earliest when the slowest has his turn.

\subsection{Fate Points}
\label{sec:org26d5756}
Each player can have up to 3 fate points. They can be used at any point in time to change a single die roll (not just jour own) to any specific value or to add an interresting aspect to a scene (GM has veto rights). Fate points can be recovered by a characters aspect being used against them or by rolling a "and .." with the fate die on a failed check. Players should start a session with 2 fate points.


\subsection{Traits}
\label{sec:org0d92ec6}
Traits are distinguishing things about the character that allow him to break a rule of the world or the game in some way. For example with the Trait Nightvision you can just see in the dark. No rolls required. Some traits (like all magic) should come with a risk (all checks that can only be made with this trait are risky checks). They can be bought for character points, this is possible both at character creation and later in the game.

\subsection{Character Creation}
\label{sec:org9ddfc80}
Distribute 100 CP on your Attributes, Aspects and Traits.

Use the Rules under Equipment to limit your starting gear.

Character Advancement:
You may reward your players with CP (character points) for reaching milestones in the story or simply surviving the session.

\begin{itemize}
\item An attribute point costs 2 CP.
\item An Aspect point typically costs 2 CP but can vary based on how specific they are.
\item A Trait typically costs 7 CP but can vary widely. Negative traits can even have a negative price.
\end{itemize}

\subsection{Movement and Range}
\label{sec:org5f7cd53}
Sometimes it is usefull to draw maps and define distances. In a contest split the area into roughly 3-5 zones. A player can move from one zone to another each round. If one can act at a range like for example when shooting a bow one can act 1-2 zones far. 

\subsection{Items and Equipment}
\label{sec:org3aaa256}
The RV (Resource Value) of an Item determines how expensive or hard to get it is. Items also have a description and maybe special effects. Let your fantasy go wild.

The effects an item has should not exceed its RV times two in AP.

\paragraph*{Armor / Damage Reduction}
\label{sec:orgbf74a67}
There is no Armor but some aspects can act as such. If an aspect can be used in a defensive action, and this aspect has any resistant points then those effectively reduce the amount of successes of the attack. Thus they act like armor. Think of armor items having the protective aspect on them.

\paragraph*{Equipment}
\label{sec:orgdbf8919}
Characters can have gear with a value of up to half the attribute Gear in RV on them. They must be able to carry all that gear on them or if it is part of their household it must fit in their normally furnished home. Apply reason as necessary.

When out adventuring characters have all the gear that they have written down. Additionally they can be allowed to make a Gear check against the RV of what they would like to have in the moment to see if they do. The check is risky and if they fail they get the difference in damage to their Gear attribute until the end of the mission.

\paragraph*{Buying}
\label{sec:org0df5909}
Characters can buy new items with a Finances check. The check is risky. The bought item can be treated like an aspect that is created with this check. The GM does not have to let you retry on a fail.

\paragraph*{Crafting}
\label{sec:orgbed6b8a}
Characters can also build their own items. For that they need the appropriate tools and resources. The resources may be bought and have a value of the item to be build minus 1. To build the item the character needs to make a check with the items value as difficulty. If that fails the resources might be lost, depending on what they are.

\paragraph*{Gathering}
\label{sec:orgea5f2c2}
Resources can be gathered with a check and their item value as difficulty.

\paragraph*{Bribing}
\label{sec:orgb6c5168}
To Bribe someone you need to give them more than they can normally comfortably afford. This means you need more than half their finances value in successes to bribe them.

\paragraph*{Creating}
\label{sec:org9cee26a}
To create an item first give it a short description. It should make clear on what kind of actions it may give advantages or what kind of effects may be created with it. Second you determine its value if applicable. You can treat it like an aspect. 

\begin{center}
\begin{tabular}{c|l|l}
\textbf{RV} & \textbf{Description} & \textbf{Example}\\[0pt]
\hline
0 & Free & a club\\[0pt]
1 & Cheap & simple clothes, basic tools\\[0pt]
2 & Affordable & regular car, apartment\\[0pt]
3 & Costly & regular house\\[0pt]
4 & Expensive & sports car\\[0pt]
5 & Very Expensive & small airplane\\[0pt]
6 & Luxurious & private jet\\[0pt]
\end{tabular}
\end{center}

\section{Optional Rules}
\label{sec:orgcdf80ff}
\subsection{Magic}
\label{sec:org7fd2c96}
Magic naratively gives a huge flexibility to explain aspects. To balance this out any checks made using magic should be considered risky. This means the value of the created aspects has to be defined beforehand. If it is not reached the magician creates an unwanted likely negative aspect at the value of the difficulty. If he succeds the created has exactly the predefined value. Depending on the setting a trait might be necessary to cast magic or even a specific kind of magic.


\subsection{Less precise Attributes}
\label{sec:orge390d39}
Instead of using the Attributes as listed you can use only the Categories. Learning a level in one of the categories costs double of what a level in an attribute would cost.

\subsection{No Abstraction for Money}
\label{sec:org0ae51e2}
To remove the resources category from the attributes just raise the price of learning a level of the other attributes by 33\%. The costs for goods and services depend on the kampaign setting.

\subsection{Retroactive Actions}
\label{sec:org24422c4}
The GM may allow players retroactively having performed some action. For example having placed a trap beforehand. To balance this any check on such an action should be a risky check.


\newpage
\section{Lists}
\label{sec:org4197b68}
None of the following lists is exhaustive. They should be taken as examples. You are invited to design your own with your group.

\subsection{List of NPCs}
\label{sec:orgedf4049}
\begin{quote}
\textbf{Average Citizen} ( CP)
Ph:3, Me:3, So:3, Re:3
\end{quote}

\begin{quote}
\textbf{Goblin} ( CP)
Ph:2, Me:1, So:1, Re:1, Nightvision
\end{quote}

\begin{quote}
\textbf{Ratling} ( CP)
Ph:1, Me:1, So:1, Re:1, Strength in numbers 1
\end{quote}

\begin{quote}
\textbf{Wolf} ( CP)
Ph:3, Me:1, So:2, Re:0, Endless endurance 1
\end{quote}

\begin{quote}
\textbf{Guard} ( CP)
Ph:4, Me:3, So:3, Re:3
\end{quote}

\begin{quote}
\textbf{Dark Mage} ( CP)
Ph:3, Me:8, So:4, Re:6, Necromancer 3, Telepathic Link to undead servants
\end{quote}

\begin{quote}
\textbf{Ogre} ( CP)
S:16, D:8, W:6, I:2, E:2 ,C:2, G:1, F:1
\end{quote}

\begin{quote}
\textbf{Zombie} ( CP)
Ph:2, Me:1, So:1, Re:1, Infectious Bite
\end{quote}

\begin{quote}
\textbf{Bandit} ( CP)
Ph: 4, Me: 3, So: 2, Re: 2
\end{quote}

\begin{quote}
\textbf{Combat Drone} ( CP)
Ph: 3, Me: 1, So: 1, Re: 1, Shooting 4, Night-vision
\end{quote}

\begin{quote}
\textbf{Orc Veteran} ( CP)
Ph: 6, Me: 4, So: 3, Re: 3, Nightvision, Reckless and Bold 2
\end{quote}

\begin{quote}
\textbf{Orc Warrior} ( CP)
Ph: 4, Me: 2, So: 1, Re: 2, Nightvision, Reckless and Bold 1
\end{quote}

\begin{quote}
\textbf{Giant Spider} ( CP)
S: 2, D: 4, W:2, I:2, E:1, C:1, G:2, F:1, Nightvision, Spider Webs 2
\end{quote}


\newpage

\section{Advice}
\label{sec:org0ee9744}
\subsection{Character Creation}
\label{sec:org29f1406}

When creating a character you should adhere to the following advice:
\begin{itemize}
\item No attribute above 6
\item No attribute below 2
\item Have 1 aspect describing what you want to be good at
\item Have 1 aspect describing what you live of
\item Have 1 aspect describing what you like to do as a hobby
\item Have at least 1 trait
\end{itemize}
Break these rules as you like.


\subsection{Encounter Design}
\label{sec:orgb794839}

To design a good challenge count the total of attibutes and aspects the players could bring to the fight and match this with those of the opponents roughly 1:1. Let your players become creative and create aspects to help them better their odds.

Try to give any noteworthy opponent an advantageous and a disadvantageous aspect.

\section{Game-play Examples}
\label{sec:org2b4b8dc}
\subsection{Character Build: Alchemist}
\label{sec:orgde960cc}

\begin{quote}
\textbf{Anna the Alchemist}

Strength: 3
Dexterity: 3
Will: 4
Intelect: 6
Empathy: 4
Charisma: 3
Gear: 6
Finances: 6

Traits:
\begin{itemize}
\item Magical Alchemy
\end{itemize}

Aspects:
\begin{itemize}
\item Third daughter of an Aristogratic Family 2
\item Proud member of the Alchemists Guild of Mistwater 3
\item Hobby Horste Rider and Trainer 1
\end{itemize}
\end{quote}

\subsection{Character Build: Babarian}
\label{sec:org66f0655}

\begin{quote}
\textbf{Bob the Barbarian}

Strength: 6
Dexterity: 5
Will: 4
Intelect: 3
Empathy: 2
Charisma: 4
Gear: 2
Finances: 2

Traits:
\begin{itemize}
\item Cold Restistance
\item Plot Armor: Can prevent getting a damaging aspect up to one time per scene.
\end{itemize}

Aspcets:
\begin{itemize}
\item Member of the isolated Nomads of the eastern steppes 2
\item Best Fighter of his tribe and wrestling champion 3
\item Gambler 1
\end{itemize}
\end{quote}

\subsection{Character Build: Generic Citizen}
\label{sec:orgaa01aeb}

\begin{quote}
Strength: 3
Dexterity: 3
Will: 3
Intelect: 3
Empathy: 3
Charisma: 3
Gear: 3
Finances: 3

Traits:
none

Aspects:
none
\end{quote}

\subsection{Character Build: Shapeshifting Durid}
\label{sec:orgecbe4a4}

\begin{quote}
Strength: 4
Dexterity: 4
Will: 5
Intelect: 3
Empathy: 3
Charisma: 4
Gear: 3
Finances: 2

Traits:
\begin{itemize}
\item druidic magic
\item Magical alchemy
\item Shapeshifting
\item Seeer
\end{itemize}

Aspects:
\begin{itemize}
\item Shapeshifting Durid 4
\item Protector of the Ancient Grove 3
\item Konwlegable in the alechemy of the gifts of nature 2
\end{itemize}
\end{quote}

\subsection{Character Build: Space Pirate}
\label{sec:orgeaa506c}

\begin{quote}
Strength: 3
Dexterity: 3
Will: 4
Intelect: 5
Empathy: 3
Charisma: 4
Gear: 3
Finances: 4

Traits:
\begin{itemize}
\item Bionic Eye with super zoom level and infrared vision. (7 CP)
\item Bionic Leg (1 CP)
\end{itemize}

Aspects:
\begin{itemize}
\item Has lift in space all his life 2
\item If the captain ordered it it has to be done 2
\item Space engineer 1
\item Gambler 2
\item Realy good with the needle 1
\end{itemize}
\end{quote}

\subsection{Character Build:}
\label{sec:org798fd8e}

\subsection{Ambushed by Goblins}
\label{sec:org64b17f9}

\subsection{Hacker duel}
\label{sec:orga1e4611}

\subsection{Court Case}
\label{sec:org4133f03}




\begin{small} This product is licensed under the ORC License held in the License of Congress at TX000 [number tbd] and available online at various locations including www.chaosium.com/orclicense, www.azoralaw.com/orclicense, www.gencon.com/orclicense and others. All warranties are disclaimed as set forth therein. This product is the original work of Lukas Zumvorde. If you use my ORC Content, please also credit me. \end{small}
\end{document}